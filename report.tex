\documentclass[12pt]{article}

\usepackage{sbc-template}

\usepackage{graphicx,url}

%\usepackage[brazil]{babel}   
\usepackage[latin1]{inputenc}  

     
\sloppy

\title{Controle para basculante}

\author{Gilson da Rosa Webber\inst{1}, Rafael da Fonte Lopes da Silva\inst{1}}


\address{Instituto de Inform�tica -- Universidade Federal do Rio Grande do Sul
  (UFRGS)\\
  Caixa Postal 15.064 -- 91.501-970 -- Porto Alegre -- RS -- Brazil
  \email{gilson.webber@inf.ufrgs.br, rflsilva@inf.ufrgs.br}
}

\begin{document} 

\maketitle

\begin{resumo}
blablabla
\end{resumo}

\section{Proposta}
Nossa proposta � a de construir um dispositivo microcontrolado com um \textbf{PIC-16F684} da $MicroChip^{TM}$ capaz de gerenciar a abertura de uma janela do tipo basculante. Desejamos projet�-lo de modo que opere de acordo com os seguintes poss�veis crit�rios:

\begin{itemize}
\item Controle manual da abertura, feito pelo usu�rio;
\item Controle autom�tico, definido de acordo com o grau de ilumina��o do ambiente onde estiver posicionado um sensor.
\end{itemize}


\subsection{Esquem�tico}





%\bibliographystyle{sbc}
%\bibliography{sbc-template}

\end{document}
