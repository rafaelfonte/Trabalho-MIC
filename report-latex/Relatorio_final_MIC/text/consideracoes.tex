\section{Considera��es finais}
A escolha da ideia do projeto nos permitiu explorar uma �rea consider�vel de problemas/solu��es de projetos eletr�nicos microcontrolados. Tivemos a oportunidade de trabalhar, testar e refinar nossa solu��o ao longo do projeto, o que trouxe um resultado muito interessante ao final.

As quest�es de configura��o do PIC, apesar de um tanto complicadas no come�o, aos poucos foram sendo resolvidas. Foi muito interessante ver como o projeto se desenvolveu enquanto agreg�vamos ideias que permitiram melhor configurabilidade do dispositivo pelo usu�rio, e isso motivou a ideia de se criar o cen�rio de exemplo para teste da solu��o obtida. Aos poucos pudemos nos acostumar com o estilo de programa��o do dispositivo, que � bem mais restritivo do que aquele ao qual est�vamos habituados do microcontrolador 8051 da Intel.

Talvez um pr�ximo passo possa ser integrar um sistema semelhante a um cen�rio real, onde os princ�pios absorvidos com o desenvolvimento de nosso sistema possa ser �til (tanto que utilizamos um problema real de um componente do grupo como motiva��o para a escolha do projeto). Outros itens que podem ser considerados em itera��es futuras podem envolver, por exemplo, a integra��o de controle de tempo no sistema (tanto cronol�gico quanto meteorol�gico). Isso poderia permitir um ajuste do grau de luminosidade desejado pelo usu�rio de acordo com o hor�rio e as condi��es do tempo medidos pelo PIC.